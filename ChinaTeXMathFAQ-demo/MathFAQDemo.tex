% !Mode:: "TeX:UTF-8"
\documentclass{article}
\usepackage[xetex]{geometry}
\geometry{
    a4paper,
   left=35mm,  %% or inner=23mm
   right=35mm, %% or outer=18mm
   top=23mm, bottom=23mm,
   headheight=2.17cm,
   headsep=4mm,
   footskip=8mm
}
\usepackage{xspace}
\usepackage{fancyhdr}
\usepackage{tipa,metalogo,paralist}
\usepackage{array}

\usepackage{amsmath,nicefrac,units}
\usepackage{ntheorem}

\usepackage{empheq}
\usepackage{harpoon}
\usepackage{delarray}
\usepackage[titletoc]{appendix}
\usepackage{graphicx}
\usepackage{multirow}
\usepackage{texnames}

\usepackage{amssymb}
\usepackage{mathtools}
\usepackage{yhmath}
\usepackage{upgreek}
\usepackage{amsfonts,mathrsfs,bm}
\usepackage{textcomp}
\usepackage[table,svgnames]{xcolor}

\usepackage{framed}
\usepackage{wallpaper}
\usepackage{xeCJK,CJKfntef}
\usepackage{cases}
\usepackage{arydshln,amscd}
\usepackage{supertabular}
\usepackage{breqn}
\usepackage{tikz}
\usepackage{blkarray}

\newcommand*{\fancybreak}{%
  \par
  \nopagebreak\medskip\nopagebreak
  \noindent\null\hfill$*\quad*\quad*\quad$\hfill\null\par
  \nopagebreak\medskip\pagebreak[0]%
}
%%%%%%%%%%%中文定制%%%%%%%%%%%
\defaultfontfeatures{Mapping=tex-text}
%\setCJKmainfont[BoldFont=SimHei, ItalicFont=KaiTi]{SimSun}
\setCJKmainfont[BoldFont=Adobe Heiti Std, ItalicFont=Adobe Kaiti Std]{Adobe Song Std}
\setCJKmonofont[Scale=0.9]{Adobe Kaiti Std}
\setCJKfamilyfont{kai}[BoldFont=Adobe Heiti Std]{Adobe Kaiti Std}
\setCJKfamilyfont{hei}[BoldFont=Adobe Heiti Std]{Adobe Heiti Std}
\setCJKfamilyfont{song}[BoldFont=Droid Sans Fallback]{AR PL UMing CN}
\setCJKfamilyfont{sf}[BoldFont=Droid Sans Fallback]{Droid Sans Fallback}

 \definecolor{darkgrey}{cmyk}{0,0,0,0.63}%{0,0,0,63}
 \definecolor{lightgrey}{cmyk}{0,0,0,0.32}%{0,0,0,32}
\makeatletter
\fancypagestyle{headings}{%
  \fancyhf{}
  \fancyhead[RO]{%
{\@tikzhead{\leftmark}}
  }
  \fancyhead[LE]{%
{\@tikzhead{\leftmark}}
  }
\fancyfoot[LO,RE]{\textcolor[rgb]{0.00390625, 0.43359375,0.77734375} {China\TeX{}} {\fontspec{Exmouth} Math {\color{red}F}A\color{lightblue}Q} $\big\vert$ China\TeX{} 倾情制作}
  \fancyfoot[RO,LE]{\tikz[baseline]\node[pagefooter]{\thepage};}
  \renewcommand{\headrulewidth}{0pt}
  \renewcommand{\footrulewidth}{0pt}
}
\fancypagestyle{headcover}{%
  \fancyhf{}
  \fancyhead[RO]{%
\@tikzhead{\leftmark}
  }
  \fancyhead[LE]{%
{\@tikzhead{\leftmark}}
  }
\fancyfoot[LO,RE]{}
  \fancyfoot[RO,LE]{}
  \renewcommand{\headrulewidth}{0pt}
  \renewcommand{\footrulewidth}{0pt}
}
\newlength\pagenumwidth
\settowidth{\pagenumwidth}{99}
\tikzset{pagefooter/.style={
anchor=base,font=\sffamily\bfseries\small,
text=white,fill=gray!45,text centered,
text depth=17mm,text width=\pagenumwidth}}
\newcommand*{\@tikzhead}[1]%
{%
  \begin{tikzpicture}[remember picture,overlay]%
    \node[yshift=-1.3cm] at (current page.north west)%
    {%
      \begin{tikzpicture}[remember picture, overlay]%
      \draw[fill=lightgrey!40,line width=0mm,draw=none](0,0) rectangle (\paperwidth,13mm);%
        \node[anchor=east,xshift=.52\paperwidth,yshift=3.5mm,rectangle,line width=0mm,draw=none]%
              {\begin{minipage}{9cm}\flushleft{{#1}}\end{minipage}};%
        \node[anchor=east,xshift=.92\paperwidth,yshift=3.5mm,rectangle,line width=0mm,draw=none]%
              {{\color{black}\textsc{China\TeX{} Documentation Workshop}}};%
      \end{tikzpicture}%
    };%
  \end{tikzpicture}%
}%
\pagestyle{headings}
\newcommand*{\Headline}[1]{\@mkboth{#1}{#1}}%
%
\newcommand{\headfont}{\normalfont\mdseries\scshape}
\makeatother

\definecolor{hellmagenta}{rgb}{1,0.75,0.9}
\definecolor{hellcyan}{rgb}{0.75,1,0.9}
\definecolor{hellgelb}{rgb}{1,1,0.8}
\definecolor{colKeys}{rgb}{0,0,1}
\definecolor{colIdentifier}{rgb}{0,0,0}
\definecolor{colComments}{rgb}{1,0,0}
\definecolor{colString}{rgb}{0,0.5,0}
\definecolor{darkyellow}{rgb}{1,0.9,0}
\definecolor{lightblue}{rgb}{.2,.5,1}
\usepackage{graphics,boites}
\usepackage{fancyvrb}

\usepackage{pifont}
\definecolor{niceblue}{rgb}{0,0.4,0.7}
\definecolor{preblue}{rgb}{0,0,0.4}
\definecolor{niceblue1}{rgb}{0.17,0.78,0.58}

%%%%%小节样式%%%%%%%%%%%%%%%%%%%
\usepackage{titlesec,titletoc}
\renewcommand\thesubsection{\arabic{subsection}}
\renewcommand\thesection{\Roman{section}}
\titleformat{\section}[display]{\normalfont}
    {\filcenter\Large\bfseries --\enspace \thesection
     \enspace--}{6pt}{\Large\bfseries\filcenter}
\titleformat{\subsection}[hang]
{\normalfont
\bfseries\sffamily\filright}
{\color{blue}\ding{45}\ \thesubsection}
{.5em}
{}[\titlerule{\hbox{\vrule height 3pt}}\vskip -.5em]



%%%%%定制代码输出%%%%%%%%%%%%%%%%%%%
\definecolor{lstbgcolor}{rgb}{0.9,0.9,0.9}
\usepackage{listings}
\lstloadlanguages{[LaTeX]TeX}
\usepackage{fancyvrb}
\newenvironment{latexample}[1][language={[LaTeX]TeX}]
{\lstset{backgroundcolor=\color{lstbgcolor},
    keywordstyle=\color[rgb]{0,0,1},
    commentstyle=\color[rgb]{0.133,0.545,0.133},
    stringstyle=\color[rgb]{0.627,0.126,0.941},
    breaklines=true,
    prebreak = \raisebox{0ex}[0ex][0ex]{\ensuremath{\hookleftarrow}},
    frame=single,
    language={[LaTeX]TeX},
    basicstyle=\footnotesize\ttfamily, #1}
  \VerbatimEnvironment\begin{VerbatimOut}{latexample.verb.out}}
  {\end{VerbatimOut}\par\noindent
  \begin{minipage}{0.5\linewidth}
    \lstinputlisting[]{latexample.verb.out}%
  \end{minipage}\qquad
  \begin{minipage}{0.4\linewidth}
    \input{latexample.verb.out}
  \end{minipage}\\
}
\makeatletter
\renewcommand\tableofcontents{%
    \centerline{\Large\bfseries \contentsname
        \@mkboth{%
           \MakeUppercase\contentsname}{\MakeUppercase\contentsname}}%
           \vskip 5.3ex%
    \@starttoc{toc}%
    }
\DeclareRobustCommand{\AllTeX}{(L%
        {\setbox0\hbox{T}%
         \setbox\@tempboxa\hbox{$\m@th$%
                                \csname S@\f@size\endcsname
                                \fontsize\sf@size\z@
                                \math@fontsfalse\selectfont
                                A}%
         \@tempdima\ht0
         \advance\@tempdima-\ht\@tempboxa
         \@tempdima\strip@pt\fontdimen1\font\@tempdima
         \advance\@tempdima-.36em
         \kern\@tempdima
         \vbox to\ht0{\box\@tempboxa
                      \vss}%
        }\kern-.075em)%
        \kern-.075em\TeX}
\makeatother


\usepackage{manfnt,amscd}

\usepackage{extarrows}
\definecolor{qianhong}{rgb}{1, 0.6, 0.75}
\newcounter{notectr}[subsection]
\newenvironment{note}
{\par\addtocounter{notectr}{1}\setlength{\parindent}{0pt}\hskip -\leftmargin {\color{gray!30}\rule{3cm}{3pt}}\vskip-5pt
{\fboxsep1pt\hskip -\leftmargin \itshape{\color{qianhong}\lhdbend} \ \colorbox{qianhong}{小注\ \thenotectr}}%
\vskip 1pt\small\itshape}
{\vskip -15pt\noindent\color{gray!30}\rule{\linewidth}{3pt}}

\definecolor{hypercolor}{rgb}{0.5,0.0,0.5}
\usepackage[xetex,colorlinks,linkcolor=hypercolor,citecolor=hypercolor,
             pdftitle=ChinaTeX Math FAQ,
             pdfauthor=ChinaTeX,
             urlcolor=hypercolor,filecolor=hypercolor,
             bookmarksopen]{hyperref}
\def\CHfaqversion{1.0}
\def\betaversion{beta}
\begin{document}
  \fontsize{12}{20}\selectfont
  \parindent 2em
  \definecolor{fondpaille}{cmyk}{0,0,0.1,0}
  \definecolor{chinablue}{rgb}{0.00390625, 0.43359375,0.77734375}
\pagecolor{gray!5}
\begin{titlepage}
  \begin{tikzpicture}[remember picture,overlay]
    \shade [left color=chinablue,right color=chinablue](current page.south west) rectangle ([yshift=\paperheight,xshift=-6.5cm]current page.center);
   % \shade[top color=black,bottom color=MyDarkRed]([yshift=7cm]current page.east)rectangle([yshift=2.5cm]current page.west);
    \node[text width=.95\textwidth,opacity=.15,yshift=0.5cm,xshift=1.5cm] at (current page.center) {%
\begin{verbatim}
\newcommand{\BigFig}[1]{\parbox{12pt}{\Huge #1}}
\newcommand{\BigZero}{\BigFig{0}}
\begin{equation*}
(a_{k1})=\left(
\begin{matrix}
0\ldots 0 & 1 & 0\ldots 0\\
& 0\\
\BigZero & \cdots & \BigZero\\
& 0\\
\end{matrix}
\right)
\qquad
f(x)=
\begin{cases}
-x^{2},
&\text{if $x < 0$;}\\
\alpha + x,
&\text{if $0 \leq x \leq 1$;}\\
x^{2},
&\text{otherwise.}
\end{cases}
\end{equation*}
\vskip 1cm
\[
\begin{CD}
A @>\log>> B @>>\text{bottom}> C @= D @<<<
E @<<< F\\
@V\text{one-one}VV @. @AA\text{onto}A @|\\
X @= Y @>>> Z @>>> U\\
@A\beta AA @AA\gamma A @VVV @VVV\\
D @>\alpha>> E @>>> H @. I\\
\end{CD}
\]
\end{verbatim}%
};
  \end{tikzpicture}
\begin{minipage}[t]{\textwidth}
\hfill \textcolor[rgb]{0.00390625, 0.43359375,0.77734375}{China\TeX{}} 文档工作室
\end{minipage}
\vspace*{3.0cm}
%\fontspec{Burgues Script} \fontspec{Loki Cola}
\pdfbookmark[1]{China\TeX{}数学排版常见问题集}{anchor}
\begin{center}
\textbf{\Huge \textcolor[rgb]{0.00390625, 0.43359375,0.77734375}{\qquad China\TeX{}}数学排版常见问题集}\\[\baselineskip]
\Huge \textbf{\textcolor[rgb]{0.00390625, 0.43359375,0.77734375}{\qquad China\TeX{}} \fontspec{Exmouth} Math {\color{red}F}A\color{lightblue}Q} Demo
\vskip 3cm
\end{center}
\newcommand{\BigFig}[1]{\parbox{12pt}{\Huge #1}}
\newcommand{\BigZero}{\BigFig{0}}
\begin{equation*}
\qquad(a_{k1})=\left(
\begin{matrix}
0\ldots 0 & 1 & 0\ldots 0\\
& 0\\
\BigZero & \cdots & \BigZero\\
& 0\\
\end{matrix}
\right)
\qquad
f(x)=
\begin{cases}
-x^{2},
&\text{if $x < 0$;}\\
\alpha + x,
&\text{if $0 \leq x \leq 1$;}\\
x^{2},
&\text{otherwise.}
\end{cases}
\end{equation*}

\vskip 1cm

\[
\qquad\quad\begin{CD}
A @>\log>> B @>>\text{bottom}> C @= D \\
@V\text{one-one}VV @. @AA\text{onto}A @|\\
X @= Y @>>> Z @>>> U\\
@A\beta AA @AA\gamma A @VVV @VVV\\
D @>\alpha>> E @>>> H @. I\\
\end{CD}
\]


\vskip 3cm
\begin{flushright}
\begin{minipage}{.7\textwidth}
\flushright
\hrulefill

Released by China\TeX\ Documentation Workshop.\\
June , 2011\\
Maker: China\TeX , Clark Ma\\
ID: chinatex, Clark\_Ma\\
\hrulefill
\end{minipage}
\end{flushright}

\end{titlepage}
\pagecolor{white}
\Headline{写在前面}
\begin{center}
{\Large\CJKfamily{hei} 写在前面}
\end{center}
\phantomsection
\addcontentsline{toc}{section}{写在前面}

\bigskip
许多网友,看了数学常见问题的说明文档,想看看其源代码。由于时间仓促,我就缩减出来了这个版本,原版本中有些处理可能阅读起来比较困难。

这个版本的内容基本保留了源文档的绝大多数的内容,原来文档代码输出使用了 \verb|minted| 包,需要第三方程序支持,在这个版本里去掉了,省去了大家配置程序的工作。

另外,代码使用 \verb|xelatex| 来编译,所需字体可从 \url{http://ftp.chinatex.org/Fonts/ChineseFonts.rar}里下载。

自己若是有兴趣可以自己添加其他更多功能成为自己的常用包,Happy \LaTeX ing!
\bigskip

\begin{flushright}
2011年7月\\
chiantex
\end{flushright}
\newpage
\pagenumbering{Roman}
\renewcommand\contentsname{目录}
\phantomsection
\pdfbookmark[1]{目录}{anchor1}
\tableofcontents
\newpage

\section[(La\kern-.075em)\kern-.075em\TeX{}数学排版如何入门?]{\AllTeX{} 数学排版如何入门?}
\AllTeX{}以其优异的数学排版能力而闻名遐迩,也是目前世界上公认排版数学公式最为优秀的
系统。对于数学排版入门,需要如何做?我们自己组织了些问题,或许能帮助您。


\subsection{我该读什么书?}
 书是人类的朋友, 学习\TeX{}公式排版,网络已有的资源非常之丰富,且
 都是十分优秀的作品。细细读来,别有韵味。

这里简单介绍些电子书资源,其简介和说明均是个人浅见,欢迎指正。

 %\renewcommand\labelitemi{\ensuremath{\blacksquare}}
 \begin{compactitem}
   \item 《More Math into \LaTeX 》,这本书洋洋洒洒写了六百多页,去掉非数学排版的部分,也有近三百页的内容,从公式的基本元素的输入到复杂公式的输入,
       逐层深入,抽丝剥茧,娓娓道来。这是我首推的一本入门书,这本书有配套视频,当然以我目前的英文水平实在
       是听不懂,若是您有兴趣可以去下载观瞻观瞻\footnote{\url{http://www.ctan.org/tex-archive/info/examples/Math_into_LaTeX-4}}。
   \item 《Math mode》,这本书是我的入门书。当然,有个人感情在里面,一直保存着,当然最近这个文档已经更新到了 2.47 版本,可见作者还是对这本书情有独钟的。我觉得他里面介绍相对上一本书要精细要深入一点,也仅仅是我个人观点。不管做怎么说都是
       吐血推荐的好书。
   \item 《\LaTeX{} Companion》Ch8,如果说高老头\TeX{}的书是论语,那么这本书算是一本史记,全面而精妙,是所有\LaTeX{} 书中的精品,当然其数学部分--Higher Mathematics,也值得拜读一下。
 \end{compactitem}
其他书籍,如《short-math-guide》、《InlineMath》、《The \LaTeX{} Mathematics Companion》、amsmath的相关说明文档等等均需看看。

\fancybreak

\subsection{我该怎么读书?}
对于读书,但凡学习 \AllTeX{} ,很多时候需要我们去阅读相关电子书,有时也需要利用搜索引擎去搜索相关问题,实际从很多学习者经验来说,我们遇到的很多问题,
在书中都已经给出了解答。往往很多初学者总是缘木求鱼,舍本逐末,去网络折腾半天,有时还找不到很恰当的答案。

第一,认真研读一本书。基本上,但凡能称得起一本书,其内容都会覆盖到我们所需的基本知识。这一步很重要。因为很多用户入门时不愿读书,
记住,\AllTeX{} 不欢迎临时抱佛脚的莽撞汉。
第二,亲自输入代码上机实验。建议初学者亲自输入代码,而不是拷贝电子书的代码来运行。
第三,材料输入,就是自己找一个公式较多的书籍,或者就是自己的论文,对照着一一输入。做这一步需要初学者能掌握一些基本的知识。
第四,实践中扩展知识,这是比较高级的阶段了,首先,基本的公式自己可以输入,诸如多行公式,复杂矩阵等,这时需要更多地思考,比如equarray,align这些
环境有哪些不同,使用上有哪些差异,我应该怎么调节公式才能得到更美观的公式等。

多多练习才是学习 \AllTeX{} 公式排版的王道。



\fancybreak
\subsection{我需具备哪些基础知识?}

由于我们这个手册并非入门的书,我们首先简单介绍下基础知识,粗枝大叶而不是面面俱到,仅作为我们手册的前奏。具体知识
大家还是要去各个电子书去逐步学习。
\begin{enumerate}
\item 输入环境;
\LaTeX{}提供了两种输入数学公式的模式:行内(inline)模式和特显(display)模式\footnote{也有翻译成\textbf{展示模式}的,但是这个词已经有了较好的译法,叫\textbf{展示模式}有些词不达意。}。前者是在\$$\cdots$\$或者\verb|\(| $\cdots$ \verb|\)|之间输入公式\footnote{行内公式的后一种输入方式其实源于\AMSTEX 的输入传统,下面的特显模式的后一种输入方式同样来源于此。},后者是在\$\$$\cdots$\$\$或者\verb|\[|$\cdots$\verb|\]|之间输入。针对于特显模式使用\verb|\begin{equation}| \ldots\verb|\end{equation}|会生成带编号公式,如不需编号,那么使用\verb|\begin{equation*}|\ldots \verb|\end{equation}|。
\item 能够输入的字符;下面的字符不能使用\footnote{这些字符在\TeX{} 中已经被定义用来表示特定意义的语法标志。}:\verb|#$%&~_^\{}|。
如果想输入上述的\verb|#$%&_{}|,
请使用这种输入方法:\verb|\# \$ \% \& \_ \{ \}|。
还有一个问题是,在数学模式中输入中文会报错,这个时候如果是行内公式的话尽量把汉字弄出\$$\cdots$\$或者\verb|\(| $\cdots$ \verb|\)|来就可以,但是如果是在特显模式情况下输入汉字,根本没有办法跳脱出来,那么请使用盒子来输入中文比如说
\verb|\mbox{中文输入}|。
\item 上标和下标;
用\verb|^|来表示上标,用\verb|_|来表示下标。如:$C^3_5$需要写作\verb|C^3_5|。
\item 希腊字母;
能够输入的希腊字母表如下:
\begin{center}
\begin{tabular}{cccccccc}
$\alpha$      & \verb|\alpha|      & $\theta$    & \verb|\theta|        &
        $o$         & \verb|o|        & $\tau$     & \verb|\tau| \\
    $\beta$       & \verb|\beta|       & $\vartheta$ & \verb|\vartheta| &
        $\pi$       & \verb|\pi|      & $\upsilon$ & \verb|\upsilon| \\
    $\gamma$      & \verb|\gamma|      & $\iota$     & \verb|\iota|     &
        $\varpi$    & \verb|\varpi|   & $\phi$     & \verb|\phi| \\
    $\delta$      & \verb|\delta|      & $\kappa$    & \verb|\kappa|    &
        $\rho$      & \verb|\rho|     & $\varphi$  & \verb|\varphi| \\
    $\epsilon$    & \verb|\epsilon|    & $\lambda$   & \verb|\lambda|   &
        $\varrho$   & \verb|\varrho|  & $\chi$     & \verb|\chi| \\
    $\varepsilon$ & \verb|\varepsilon| & $\mu$       & \verb|\mu|       &
        $\sigma$    & \verb|\sigma|   & $\psi$     & \verb|\psi| \\
    $\zeta$       & \verb|\zeta|       & $\nu$       & \verb|\nu|       &
        $\varsigma$ & \verb|\varsigma|   & $\omega$   & \verb|\omega| \\
    $\eta$        & \verb|\eta|        & $\xi$       & \verb|\xi|       &
        &                 &            & \\
    $\Gamma$      & \verb|\Gamma|      & $\Lambda$   & \verb|\Lambda|   &
        $\Sigma$    & \verb|\Sigma|   & $\Psi$     & \verb|\Psi| \\
    $\Delta$      & \verb|\Delta|      & $\Xi$       & \verb|\Xi|       &
        $\Upsilon$  & \verb|\Upsilon| & $\Omega$   & \verb|\Omega| \\
    $\Theta$      & \verb|\Theta|      & $\Pi$       & \verb|\Pi|       &
        $\Phi$      & \verb|\Phi|     &            & \\
\end{tabular}
\end{center}
%\item 函数;
\item 分数与开方;分数用\verb|\frac{分子}{分母}|,开方用\verb|\sqrt[n]{表达式}|。如
\begin{latexample}[]
$\frac{1}{\pi}$,\quad
$\sqrt[5]{1+k^2+k^4}$
\end{latexample}
\noindent 在输入根式是n省略的情况下会默认为开平方模式。
\item 省略号;
使用下面的输入来输入不同的省略号:
\begin{center}
\begin{tabular}{cccc}
$\dots$ & $\cdots$ & $\vdots$ & $\ddots$\\
\verb|\dots| & \verb|\cdots| & \verb|\vdots| & \verb|\ddots|
\end{tabular}
\end{center}
\item 括号和分隔符;
()和[ ]和\textbar{}都可以直接输入,\verb|{}|对应于要输入 {\ttfamily
\char'134\{\char'134\}},而双线 $||$ 要使用\textbackslash \textbar{}输入。当要显示大号的括号或分隔符时,要对应用\verb|\left|和\verb|\right|,如输入:
\begin{latexample}[]
\[f(x,y,z)  = 3y^2 z \left(3 +
\frac{7x+5}{1+y^2}\right).\]
\end{latexample}

要注意,\verb|\left|和\verb|\right|只是用来匹配的,本身并不显示,如:
\begin{latexample}[]
\[\left.\frac{du}{dx}\right|_{x=0}\]
\end{latexample}
\item 多行公式;如下的多行公式:
\begin{latexample}[]
\begin{eqnarray*}
\cos 2\theta &=& \cos^2 \theta - \sin^2 \theta\\
             &=& 2 \cos^2 \theta - 1.
\end{eqnarray*}
\end{latexample}

其中\&是对其点,表示在此对齐。*使\LaTeX{}不自动显示序号,如果想让\LaTeX{}自动标上序号,则把*去掉

\item 矩阵;如下矩阵
\begin{latexample}[]
\[ \left(
\begin{array}{ccc}
a & b & c \\
d & e & f \\
g & h & i
\end{array}
\right)\]
\end{latexample}
和
\begin{latexample}[]
\[ \chi(\lambda) = \left|
\begin{array}{ccc}
\lambda - a & -b & -c \\
-d & \lambda - e & -f \\
-g & -h & \lambda - i
\end{array} \right|.\]
\end{latexample}

\item 导数、极限、求和、积分。下面的微商式:
\begin{latexample}[]
\[\frac{du}{dt}\mbox{和}\frac{d^2 u}{dx^2}\]
\end{latexample}


下面的热方程:
\begin{latexample}[]
\[ \frac{\partial u}{\partial t}
   = h^2 \left( \frac{\partial^2 u}
              {\partial x^2}
   + \frac{\partial^2 u}{\partial y^2}
   + \frac{\partial^2 u}{\partial z^2}\right)
\]
\end{latexample}
下面的一些符号:
\begin{latexample}[]
\[\lim_{x \to +\infty},\inf_{x > s},\sup_K\]
\end{latexample}
下面的极限表达式:
\begin{latexample}[]
\[
\lim_{x \to 0} \frac{3x^2 +7x^3}{x^2 +5x^4} = 3.
\]
\end{latexample}

下面的求和式:
\begin{latexample}[]
\[ \sum_{k=1}^n k^2 = \frac{1}{2} n (n+1).\]
\end{latexample}

下面的积分式:
\begin{latexample}[]
\[ \int_a^b f(x)\,dx.\]
\end{latexample}
在排版中,如$dx$需要和之前的积分式排的更紧密一些,那么要使用\verb|\,|:

\begin{latexample}[]
\[ \int_0^R \frac{2x\,dx}{1+x^2} = \log(1+R^2).\]
\end{latexample}
\begin{latexample}[]
\[ \int_0^R \frac{2x\,dx}{1+x^2} = \log(1+R^2).\]
\end{latexample}
多重积分号的积分号之间也要调整距离,使用\verb|\!|:

\begin{latexample}[]
\[ \int_0^1 \! \int_0^1 x^2 y^2\,dx\,dy.\]
\end{latexample}
\end{enumerate}

\fancybreak

\end{document}
