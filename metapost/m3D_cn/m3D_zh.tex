%%%%%%%%%%%%%%%%%  m3D 宏包手册中译本 %%%%%%%%%%%%%%%%%%%%%%%%%%%%%%
%
% 1. m3D 宏包是法国 Anthony Phan 写的一个 metapost 宏包,主要是用来
%     画三维图形,且可以做的某种程度的三维消隐,功能比较强大,为此花了 
%     一些功夫翻译的该手册,希望对喜欢用 metapost 画图的朋友有用。 
%
% 2. 中文间距的调整用的是 Yin Dian 的 zhspacing 宏包,因此编译需要
%    xetex 0.997 以后的版本,我是在 W32TeX 下编译的。又采用了 lyanry
%    的版式设计,在此一并致谢!
%
% 3. 本人的英文水平太蹩脚了,虽然尽了很大力气翻译,但仍会有很多贻笑大
%    方的译文,请帮助校对和修改。由于上网的机会很少,大家的意见可能很
%    长时间不能回复,大家也可以把意见发到我的邮箱:  
%                      cvgmt@163.com
%
% 4. 如果原作者的 m3D 更新了,这个糟糕的译本也希望能相应地更新。
%
% 5. 关于 cylinderlike 的图例我把欧拉角改了,因为原来角度的消隐好像有
%    问题,又为了使得文档不超过 512k, 我把图形的 Resolution 改了一下,
%    从而改了分辨率,又 W32TeX 好像不能 download 相应的字体,因此图形
%    中的字体有问题,于是我在 Linux 下的 TeXLive2007 在 mp 文件
%    beginfig 前加入 filenametemplate"%j-%c.eps";prologues:=3; 先生成
%    eps,  再用 epspdf 转成 pdf 嵌入的。 
%
% 6. 原文中关于欧拉角的描述是错误的(但给出的公式是正确的),有时间的
%    话,希望有人能写出一篇关于 m3D 中欧拉角介绍性小文章。 
% 
% 7. tex 文档附上,我对 tex 排版没研究,其中的排版设置请斧正。
%
%                                                    ------ cvgmt
%
%                  
%%%%%%%%%%%%%%%%%%%%%%%%%%%%%%%%%%%%%%%%%%%%%%%%%%%%%%%%%%%%%%%%%%%%%%%% 
\documentclass[a4paper,12pt]{article}
\usepackage[no-math,cm-default]{fontspec}


%%%%%%%%%%%%%%%%% Yin Dian 的中文间距宏包 %%%%%%%%%%%%%%%%%%%%%%%%%%%%  
\usepackage{zhspacing}
\usepackage[fakebold]{zhfont}
\usepackage[noactive]{zhmath}
\zhspacing
\makeatletter
\g@addto@macro\zhs@savefont{%
\long\edef\zhs@tmpmacro{\f@family}%
\def\zhs@curr@fam{0}%
\ifx\zhs@tmpmacro\sfdefault
\def\zhs@curr@fam{1}%
\else\ifx\zhs@tmpmacro\ttdefault
\def\zhs@curr@fam{2}%
\fi\fi
\edef\zhs@tmpmacro{\f@family}%
\ifx\zhs@tmpmacro\sfdefault
\def\zhs@curr@fam{1}%
\else\ifx\zhs@tmpmacro\ttdefault
\def\zhs@curr@fam{2}%
\fi\fi
}
\newfontfamily\zhrmfont[BoldFont=SimHei,
ItalicFont=KaiTi_GB2312]{SimSun}
\newfontfamily\zhsffont{SimHei}
\newfontfamily\zhttfont[BoldFont=SimHei]{KaiTi_GB2312}
\def\zhfont{\ifcase\zhs@curr@fam\zhrmfont\or\zhsffont
\or\zhttfont\else\zhrmfont\fi}

\usepackage{fancyvrb}
\DefineVerbatimEnvironment{verbatim}{Verbatim}{baselinestretch=0.9}

%%%%%%%%%%%%%%% lyanry 的部分版式设计 %%%%%%%%%%%%%%%%%%%%%%%%%%%%% 

\usepackage[top=1.3in,bottom=1.1in,left=1.2in,right=1in]{geometry}
\setlength{\parindent}{22pt}
\setlength{\parskip}{0pt}
\renewcommand{\baselinestretch}{1.38}
\usepackage{indentfirst}

\newlength{\mmfontsize}
\setlength{\mmfontsize}{11pt}

% 段落间距
\setlength{\parskip}{0pt}

% list 参数配置命令
\renewcommand{\labelitemii}{$\circ$}
\newcommand{\listparam}{\setlength{\parsep}{\parskip}
 \setlength{\itemsep}{0ex plus 0.1ex}
 \setlength{\labelwidth}{2\mmfontsize}
 \setlength{\labelsep}{.5\mmfontsize}
 \setlength{\topsep}{0pt}
 \setlength{\partopsep}{0pt}
 \setlength{\leftmargin}{2\mmfontsize}
}


\usepackage[pdfstartview=FitH,
bookmarksnumbered=true,bookmarksopen=true, colorlinks=true]{hyperref}  
\hypersetup{pdfauthor={作者:Anthony Phan 译者:cvgmt}, pdftitle={m3D 宏包手册中译本},pdfsubject={MetaPost},pdfkeywords={metapost,3D,tex,MetaFont,欧拉角,消隐,投影}}   
%%%%%%%%%%%%%%%%%%%%%%%%%%%%%  正文  %%%%%%%%%%%%%%%%%%%%%%%%%%%%%% 

\begin{document} 
\title{\Huge { M3D 宏包手册}}
\author{Anthony Phan }
\date{2006 年 12 月 18 日}
\maketitle
\section*{引论}
在九十年代(刚过去的世纪)后半段时候,我发现\TeX, MetaFont,然后是 
MetaPost。我确实着迷于最后的那个工具,因为它用一种类 MetaFont 的语言 
生成图形,且允许非常简单图形包含到\TeX 文档中去。我在 3D 图形中的 用 
MetaPost  的最初尝试很简单:投影系统是纯粹的刚体,图形是由一些线和 标
签以及仅仅 4 个平坦的面填上固定的颜色。于是我觉得必须要一个参数化的标 
架,因为我们并不总是知道一个图形是否已经从一个合适的角度观察,这或许富 
有意义。 从而最首要的步骤完成:有了一个参数化的标架,处理空间坐标,如 
果它们在屏幕的投影是定向良好的,就画出轮廓。 

然后我听说 Denis Roegel 的~``m3D'' 宏包。它的语法不适合我在 MetaPost
中已有的 3D-程序的想法,但它的特性便利于从 ~{\tt eps} 转换到 ~{\tt
  gif} 和动画。我偷学了这个动画设备,自娱以及为我的图形寻找好的角度。
为了我的网页插图, 我开始了更为复杂的设计 $\cdots\cdots$ 

我的{\it 中心思想}是尽可能保持与通常的  MetaFont 和  MetaPost 程序靠
拢,但也考虑到所有的 3-维对象如此的复杂性,应当用上一些非常有技巧的代 
码。因此我转而思考如何尽善尽美地得到一个稳定的, 功能强大的基本程序,
以及在其图库中加入的一些通常对象。 于是一些相对复杂的对象现在可以通过 
这些相对基本的对象通过移动,旋转以及放大而得到。 当然,我关于对象
的想法与面向对象-编程无关, 它只是关于刚体朴实的观念而已。

如此的计划所能达到的目标是有限的,这个~``3D'' 宏包仍在发展中。当某个朋 
友问我是否能画出一个比他画得更好的图形时, 我很高兴的是能用它去完成。

我的其中一个信念是那些 MetaPost 编程着宁愿建立他(她)的宏包系统而不去
依赖某人的程序 --- 特别是当这些程序被声明是不稳定的时候。这些编程者只
是简单地回顾一些语法, 一些特别的改进,就公布他(她)自己的程序到网上,
用这种方式对其他所有感兴趣于这个程序人给出某些反馈。

\section{基本概念}
众所周知,MetaPost 提供了标准的设备操控称为颜色的 3 维向量的变量类型:
数据类型的检查,加法, 减法,数乘。由此看来偏偏缺少了3 维向量和图形的
仿射变换,可是没有人会去抱怨这个, 因为 MetaPost 是一个2-维定向程序语
言,因此我们会接受用一些高层次的控制序列来完成这些任务。 

\section{坐标系统}
\centerline{\XeTeXpdffile m3Dmanual-1.pdf}
  
\section{欧拉角}
给出正交的标架 $(\mathbf{Ox},\mathbf{Oy},\mathbf{Oz})$  和
$(\mathbf{Ox}',\mathbf{Oy}',\mathbf{Oz}')$, 欧拉角
$(\theta,\phi,\psi)$ 是满足
$$
\left\{
  \begin{array}{lll}
  \mathbf{Ox'}&=\cos\phi\cos\theta\times\mathbf{Ox}+\cos\phi\sin\theta 
\times\mathbf{Oy}+\sin\phi\times\mathbf{Oz}\\
  \mathbf{Oy'}&=-(\cos\psi\sin\theta+\sin\psi\sin\phi\cos\theta)
\times\mathbf{Ox}\\
              &\quad +(\cos\psi\cos\theta-\sin\psi\sin\phi\sin\theta)
  \times\mathbf{Oy}+\sin\psi\cos\phi\times\mathbf{Oz}\\
  \mathbf{Oz'}&=(\sin\psi\sin\theta-\cos\psi\sin\phi\cos\theta)
  \times\mathbf{Ox}\\ 
  &\quad-(\sin\psi\cos\theta+\cos\psi\sin\phi\sin\theta)\times\mathbf{Oy}  
 +\cos\psi\cos\phi\times\mathbf{Oz}
\end{array}
  \right.
  $$
 的实数组,它们可以由方向 $\mathbf{Ox}'$ 按照
 $(\mathbf{Ox},\mathbf{Oy},\mathbf{Oz})$-坐标系表示的相关知识确定 ---
 我们得到 $\theta$ 和 $\phi$,而由 $(\mathbf{Ox}',\mathbf{Oy}')$ 生成
 的平面表示出一个通过  $(\mathbf{Ox},\mathbf{Oy},\mathbf{Oz})$ -坐标系 
 表示的落在  $(\mathbf{Ox'},\mathbf{Oy'})$-平面中的一个向量,它与
 $\mathbf{Ox}'$ 不共线 --- 我们得到最后的  $\psi$.\footnote{此段关于
   $\theta$, $\phi$ 的叙述与球面坐标相混淆了,与通常书上关于欧拉角的叙  
   述不符,请查看理论力学课本中相关的内容 --- 译注} 

   
  因此,一个称为 ~{\tt Angles} 的控制序列在 ~{\tt m3Dplain.mp} 中定义
  了。 它的参数是两个向量(或颜色), 称为 $p$ 和 $q$, 它们按下面的途
  径返回相应于欧拉角的三元组(或颜色) ,称为 $(\theta,\phi,\psi)$:
  如果   $p$ 为零或者是一个非常小的向量, 它返回 $(0,0,0)$; 否
  则,$\theta$   和 $\phi$ 按使得 $\mathbf{Ox}'$ 与 $p$ 具有相同的方向  
  计算;接着,如果 $q$ 看起来与 $p$  非常接近于共线,$\psi$ 就设为 
  $0$, 否则,$\psi$ 设置为帮助定义整个
  $(\mathbf{Ox}',\mathbf{Oy}',\mathbf{Oz}')$-正交标架合适的数值,以使
  得 $(p,q)$-向量平面等价于 $(\mathbf{Ox}',\mathbf{Oy}')$-向量平面且
  $\mathbf{Ox}'$ 与 $p$ 有同一方向的。  
  
  
\section{投影系统}

\section{演示参数}

有很多参数被 m3D 采用了,描述如下。对不建议改动的参数我们打上一个剑号
($^{\dagger}$), --- 很大程度上是为了审美这个理由 --- 在一幅图中。
\begin{list}{$\bullet$}{\listparam}
\item{\tt ObsZ:=}内部数值$^{\dagger}$.
  这是屏幕或纸张与观察者之间的距离。它可以带上长度单位。
\item~{\tt Resolution:=}内部数值$^{\dagger}$.
  一些预定义的对象用这个参数来决定画图的步骤数。它可以带上长度单位。
\item~{\tt LightSource:=}颜色$^{\dagger}$.
  它是光源的位置。
\item~{\tt LightAtInfinity:=}布尔量$^{\dagger}$.
  它指示光源是空间中真正的点还是方向。
\item~{\tt Luminosity:=}内部数值$^{\dagger}$.
  这是入射光的强度,它在 0 到 1 之间。
\item~{\tt Contrast:=}内部数值$^{\dagger}$.
  这是通常的对比度参数,它在 0 到 1 之间。
\item~{\tt Specularity:=}内部数值。
  它表示由物体反射的入射光线的比例。它在 0 到 1 之间且在同
  一图中可以改变,以便粉刷出不同的质感。             
\item~{\tt Phong:=}内部数值。
  它表示反射光线在物体展布的量(光泽),它是一个小整数,
  而且可以在同一图中改变,以便粉刷出各类不同的质感。
\item~{\tt Fog:=}内部数值$^{\dagger}$.
  “雾”可以用来粉刷物体。值为 $0$ 意味着没有雾,$1$ 为以指衰减的线性
  雾,$2$ 为以指数衰减的球状雾。                 
\item~{\tt FogHalf:=}内部数值$^{\dagger}$.
  它是雾的指数衰减率。它是一个带上度量单位的正数。
\item~{\tt FogZ:=}内部数值$^{\dagger}$.
  它是当雾还可见时相对屏幕往下的 $z$-坐标 。它可以带上度量单位。                     
\item~{\tt FinePlotFlag:=}布尔量。
  这个布尔量用于某些控制序列,比如 ~{\tt Plot3D}。
\item~{\tt mthreeDfont:=}字符串$^{\dagger}$.
  用于3 维空间中的文本的字体名称。
\item~{\tt ShadedTextFlag:=}布尔量。
  这个布尔量指示了在3 维空间演示一个文本是否已经用上了特殊效果。
\item~{\tt ObjectColor:=}颜色。
  这是在空间中填充一个面用的颜色。它可以在任何时候改变,以便使得图形有  
  丰富的颜色。
\end{list}
                           
\section{关于光源}

在 ~{\tt m3Dplain.mp} 只定义了一个光源。它可以设置在无穷远(~{\tt
  LightAtInfinity:=true}) 或场景中的一些点(~{\tt
  LightAtInfinity:=false})。在所有的情形中它的坐标  ~{\tt
  LightSource } 用于整个屏幕标架且当对象平移或旋转时不会改变,也就
是对观察者来说光是固定的。 

如果希望光源和特定的对象关联,我们得在对象的中定义写上诸如
\[
\hbox{\tt LightSource:=GDir(x,y,z)}
\]
如果光源是座落在无穷远;或者
\[
\hbox{\tt LightSource:=GCoord(x,y,z)}
\]
如果光源是座落在空间中的某一点 (~{\tt (x,y,z)} 是该处的{\it 局部}坐
标) 。我们注意到这是控制序列 ~{\tt GDir}  和 ~{\tt GCoord} 的使用中非 
常仅有的方向。 

如果需要多个光源,我们得重新定义 ~{\tt m3Dplain.mp} 的控制序列~{\tt
  Light }  (由于它已经是沉重的机构,所幸能行得通)

\section{从外面,里面看物体}
通常,观察者从物体的外面看物体。因此,如果他们的面的定向轮廓的投影呈现
出正的定向,则该面要画出。 这就是为什么控制序列 ~{\tt Orientation} 默
认值是 ~{\tt Outside}. 相反的情形我们可以用
$$\hbox{\tt Inside;} $$
然后可以用
$$\hbox{\tt Outside;}$$
回到默认情形。
还可以尝试下面的
$$\hbox{\tt OutsideIn;}
\qquad\hbox{ 或}\qquad
\hbox{\tt InsideOut;}$$
在这种情形下,面总是画出。两者之间的区别是后者的光线效果恢复原状。如果  
希望达到更加奇异的效果, 他(她)可以摆弄在 ~{\tt m3Dplain.mp} 中作为基
础而写进的 ~{\tt Orientation} 定义和 ~{\tt   Orienation\_} 数值。

\section{顺序及消隐}
画单独一个凸体是一件容易的事情:只要画出所有轮廓可见的面,投影,当画它
的外部(或内部)边缘时作为一个正的(或 负的)定向路径。 因此,在这种情
况下消隐面不成问题。内和外的转换可以由 ~{\tt Inside} 和  ~{\tt
  Outside} 控制序列。它们简单地使得定向的条件反过来。  

当处理非凸体的时候,我们不得不分解这个物体为凸的几部份,然后以合适的顺
序画出。因此,我们  一个相当有技巧性控制序列称为 ~{\tt QuickSort} 已经
为这个目的而设计了。 它是一些必须包括至少两项的文字论述且它的输出是
一个称为 ~{\tt SortedList} 的控制序列 , 它的内容是先前的列表并被整理为
依照 ~{\tt SortCriterion}. ~{\tt SortCriterion} 是一个包括两个论断(将
要被分类的对象列表), 它们替代文字是一个布尔量。默认情况下,这个论断
是三元组且这个条件是关于它们实际上的相对 当前观察者的深度。因此  ~{\tt
  QuickSort({\it 三元组的列表})}  将输出 ~{\tt SortedList} ,它的替代文
字正好是期望的三元组列表顺序。 我们要想改变的话,只须改写 ~{\tt
  SortCriterion}  ,以便对数值,二元组, 字符串分类。

如果我们依照空间中点的列表的深度执行一系列动作,那么按照这个方法去分类
将会是优雅的。 一个自然的做法是采用下面的过程:

{\tt OnDepth;}

{\tt Refpoint {\it 三元组};}

{\tt Action ({\it 界定的控制序列})}


\noindent
这个工作程序如下:在 ~{\tt OnDepth} 阶段保存和重设一些东西;使~{\tt
  Action\_counter} 递增,  以及保存那些在 ~{\tt Refpoint} 援引的当前参
考点(一个{\it 三元组});保存{\it 界定的控制序列 }到一个可变的控制序
列以当前 ~{\tt Action\_counter} 计数。 (此处有一个小小的技巧我寻觅了 
很长很长一段时间);在 ~{\tt endOnDepth}  依照参考点的的深度安排
$1,\dots,~{\tt  Action\_counter}$ 这些列表, 然后依照存储的列表执行动
作。在这个进程中最为有趣的是动作可以依赖一些参数, 就像循环或宏包参数。
注意到 ~{\tt SortCriterion} 当执行 ~{\tt endOnDepth}  有一个特殊和临时
的涵义: 它的两个论断便成为一些在 $\hbox{1},\dots,  \hbox{\tt
  Action\_counter}$ 中的指示在两个对应参考点深度比较。  

\section{整合文本}
很清楚,为了得到好看和有趣的图,有时需要集成一些文句在三维空间,正如我 
曾经做过一两次让文本环绕着图形。 这是比较特别的。为此定义一些一般的方
案对我来说是无意义的:它非常复杂, 很难设想人们愿意去做。无论如何,一
些控制序列允许移动平坦的文本绕圈会是一个要素。 而且,象这样基本的程序
可以有助于设计特殊的控制序列用 于复杂的任务。

\subsection{简单文本}
一个被称为 ~{\tt simpletext} 的对象在 ~{\tt m3Dplain.mp} 定义了。它的 
特殊参数的组成,首先是一些字符串描述定位  (~{\tt"left"}, ~{\tt
  "justify"}, ~{\tt "center"},~{\tt "right"}), 然后是一些告知文本的参
考点在何处(特别是~{\tt "right"}, ~{\tt "urt"},~{\tt "top"},~{\tt
  "ulft"},~{\tt "left"},~{\tt "llft"}, ~{\tt "bot"},~{\tt "lrt"} 或
[say!]~{\tt "center"}, 那么, 一列字符串就会一个叠一个地呈现出来。 

比如,在顶部许多层将会展示(更进一步{\it 看}关于放大参数)

\begin{verbatim}
UseObject(simpletext, Origin, (90, 0, 90), 10pt, "justify", "ulft",
		"Come let me sing into your ear;",
		"Those dancing days are gone,",
		"All that silk and satin gear;",
		"Crouch upon a stone,",
		"Wrapping that foul body up",
		"In as foul a rag:",
		"I carry the sun in a golden cup;",
		"The moon in a silver bag.");
\end{verbatim}

 J.B.~Yeats 的诗~ ``Those dancing days are gone'' 的第一部分,在基点
 ~{\tt Origin }上, 依照标架旋转了 $(90,0,90)$, 放大了 $10\,\rm pt$.
 如果可以的话, 每一行将会被调整,基点将参照文本的左上角。    
 
 这个 ~{\tt simpletext} 对象采用由字串 ~{\tt mthreeDfont}命名的字体(默
 认值:~{\tt "rphvb"})。 它的设计尺寸由被称为 ~{\tt  mthreeDfontsize}
 (默认值是:~{\tt 10pt}) 数值变量定义。数值参数 ~{\tt baselineskip} 按
 通常角色(默认值是:~{\tt 12pt}) 。这些参数仅仅与与字体相关,与 ~{\tt
   CurrentScale}无关。因此, 当用 ~{\tt Simpletext} 到一个对象时, 注
 意{\it 尺寸}的叙述要像
   $$
   \hbox{\tt UseObject(simpletext,{\it 原点},{\it 欧拉角},{\it 尺寸},
     {\it 列出字串},{\it 字串方位},{\it 字串列表})}
  $$
 因为{\it 尺寸}必须是一个{\it 局部尺寸}。
 
 调整是通过伸展字体的常规空间的宽度到 ~{\tt TextStretchFactor}
 得到的。(默认值 $2$ 已经相当大)。 逐次地进入句子的空间: {\it 它们
   不会缩小为一个单独的空间}。 
 
 
 每一字母分别画出只是为了 保证当投影不是线性时 ,作用在字母上的仿射变
 换近似正确 。  { 由于每一个字母有一个特殊的规模,我们必须设 ~{\tt
     prologues} 为 1 或 2  以便不会超出 MetaPost 的能力(全体字体只会
   简单地在 eps  图形的开始声明且不是每个字母采用它自己的变换)。} 当
 然,我们可以为此采用  PostScript\textsuperscript{TM} 字体。这就是为什 
 么我们已经引进  ~{\tt mthreeDfont} 缘故。
 
 
\subsection{弯曲文本}
改进中,还没有公布。

{\it 评论} --- 记住在图形注释仍然可以采用通常的控制序列就像 
$$
\hbox{\tt label{\it.方位}({\it 标签}, proj(x, y, z))}
$$
诸如 ~{\tt simpletext} 或 ~{\tt curvedtext} 是为了相当特殊的效果。


\section{动画}
\subsection{引论}
Denis Roegel 论证了用通常的 Unix 工具合并一系列 MetaPost 输出成一个
GIF 动画(3D 宏包)。 我从他的~``metapost to shell''~脚本中学了不少。
想法如下是:首先 ,跟进所有的输出的 eps 的最大边框界限;然后转换这些
eps 输出为以这个最大边框为界限的  PostScript\textsuperscript{TM}或 eps
文件;分别转换这些最后得到的文件为单一的  GIF 图像;然后再合并所有这些
图像为一个动画。  

\subsection{ 在m3D 如何做} 
在文件的末尾将会生成一个额外的脚本,它的名字默认是~{\tt animate-script}
。一旦 MetaPost  的工作完成,在类 Unix 系统面板的终端( ~{\tt
  xterm})的当前目录中执行 
$$
\hbox{\tt bash animate-script}
$$
 最后的输出是 ~{\tt jobname.gif} ,此处 ~{\tt jobname} 是 MetaPost 程
 序确实的名字。 

\subsection{额外程序}
这个脚本需要下列程序:~{\tt sed} 和 ~{\tt convert}。我们选用 ~{\tt sed}
来改变所有 MetaPost  输出的边界参数 --- 所得到的临时文件命名为 ~{\tt
  jobname.xxx.eps},此处 ~{\tt jobname.xxx} 为 MetaPost 输出文件名之一。 
这个 PostScript\textsuperscript{TM}  到 GIF 的转换由 在 Roegel 的宏包
中的 Netpbm Library 执行且它们的合并到一个动画中是由  ~{\tt
  gifmerge}(一个非标准但非常漂亮的 Unix 程序, 可以在网上自由获得)。
最近几年,ImageMagick (开始版权归  Dupont de Nemours,后来是
ImageMagick Studio,但实际上是非常自由 )已经散布到几乎所有的 Linux 发
行版本中。它是一个把任意东西转成所有东西甚至动画的高质量工具。由前面脚
本引用的其中 一个基本的控制序列是  ~{\tt convert}。


\subsection{细节}
下面提供更为详细的解释。
\begin{list}{$\bullet$}{\listparam}
\item~{\tt Animate({\it  数值},{\it 布尔量或颜色})}
\item~{\tt AnimateScipt} 字符串变量,由 ~{\tt Animate} 输出的(bash)
  脚本的名字。
\item~{\tt AnimateFormat} 字符串变量,动画像的格式。默认值是
  ~{\tt"gif"} ,但可以改变,比如,~{\tt "mpg"} 或 ~{\tt "mng"}.这个格
  式必须是 ~{\tt convert} 命令能辨认的,或其他的将可得到        的程序
  (针对 ~{\tt "mpg"} 格式的~{\tt mpg2encode})。  
\item~{\tt AnimateQuality} 数值变量,特别地可以取值为 1,2,4 \dots 
\item~{\tt AnimateDelay} 数值变量,在动画中每帖$1/100$ 秒时间。
\item~{\tt AnimateLoop} 数值变量,动画的参数,它的默认值等于 0 (无限
  次循环播放)。
\item~{\tt compute\_bbox} 以及 ~{\tt xmin\_}, ~{\tt xmax\_},~{\tt
    ymin\_}, ~{\tt ymax\_} 已经在前面解释过。 
\end{list}
  
\section{直接输出为 eps }
{\tt DirectEPS {\it 文件名};}

$\cdots\cdots$

{\tt endDirectEPS;}


\section{一些例子}
第一张图由相当于 ~{\tt TechnoFill} 的 ~{\tt Fill} 生成(我某天得换掉这个
名字),接下去的一个由相当于  ~{\tt WireFill} 的 ~{\tt Fill}生成(这
种图形相当的传统)。 我已经加入了一个文本(W.B.Yeats)来测试 ~{\tt
  simpletext} 对象。 
在第二张图中有一个圆柱 ,但更为重要的是一个称为 ~{\tt tube} 的对象的用法:通
过 $x(t), y(t), z(t)$,  一个半径 $r$, 一个 $t$ 的范围给出的空间中的一
条路径,我们可想像得出这个对象。 然而运算非常易碎(二阶)且可能导出意
料之外甚至丑陋的效果。如果对象  ~{\tt cylinder} 在 ~{\tt m3Dlib01.mp} 定
义,那么 ~{\tt tube} 在 ~{\tt   m3Dplain.mp} 就是因为我觉得它是一个基本
的工具。 

\begin{verbatim}
let Fill = SolidFill;%TechnoFill;
ShadedTextFlag := true;
TextColor := red;
beginfig(thisfig);
  interim prologues := 1;
  OnDepth;
    Refpoint(1,0,0);
    Action
    (UseObject(etube, Origin, (0, 90, 0), 1cm,
  "(cosd(t*90), 0, t)", 0.25, -3, 0, true, false));
    Refpoint(-1,0,0);
    Action
    (UseObject(etube, Origin, (0, 90, 0), 1cm,
  "(cosd(t*90), 0, t)", 0.25, 0, 3, false, true));
    for a = 0 step 45 until 315:
      Refpoint Dir(a+95,0);
      Action
      (UseObject(simpletext, Origin, (a+95, 0, 90), 5pt, "left", "ulft",
    "Come let me sing into your ear;",
    "Those dancing days are gone,",
    "All that silk and satin gear;",
    "Crouch upon a stone,",
    "Wrapping that foul body up",
    "In as foul a rag:",
    "I carry the sun in a golden cup;",
    "The moon in a silver bag."));
    endfor
  endOnDepth;
endfig;
\end{verbatim}
\centerline{\XeTeXpdffile m3Dmanual-2.pdf}


\begin{verbatim}
let Fill = WireFill;
beginfig(thisfig);
  ObjectColor := (1, 215/255, 0);
  for i = -2 upto 2:
    UseObject(tube, Origin, (0, -90, 0), 0.75cm,
      "(cosd(t*360), sind(t*360), t)", 0.25, i-0.5, i,
      if i = -2: true else: false fi, false); endfor
  ObjectColor := 0.5white;
  UseObject(cylinder, (-2.5, 0, 0)*0.75cm, (0, -90, 0), 0.75cm, 0.4, 5);
  ObjectColor := (1, 215/255, 0);
  for i = -2 upto 2:
    UseObject(tube, Origin, (0, -90, 0), 0.75cm,
      "(cosd(t*360), sind(t*360), t)", 0.25, i, i+0.5,
      if i = 2: true else: false fi, false); endfor
endfig;
\end{verbatim}
\centerline{\XeTeXpdffile m3Dmanual-3.pdf}




这里有两个 Sierpinski--Menger 对象:海绵(对象命名为 ~{\tt
  sierpinskip\_sponge}) 和垫圈(对象命名为 ~{\tt sierpinskip\_gasket})。 
这个海绵由等价于 ~{\tt SolidFill} 的 ~{\tt Fill } 画出,而垫圈由等价于
~{\tt SolidWireFill} 的 ~{\tt Fill} 画出。两个对象都在 ~{\tt
  m3Dlib01.mp} 中定义 。这个垫圈 --- 由于以 $4^{n}$ 增长,此处 $n$ 是循环
的阶数 --- 比画海绵容易得多 --- 后者以 $20^{n}$ 增长。在我当前的 MetaPost 的
实现中对海绵达到 3 阶不是一件容易的事情。 

\begin{verbatim}
ObjectColor:=0.75white;
let Fill = SolidFill;
beginfig(thisfig);
  UseObject(sierpinski_sponge, Origin, (10,0,0), 1.5cm, 2);
endfig;
\end{verbatim}
\centerline{\XeTeXpdffile m3Dmanual-4.pdf}

\begin{verbatim}
beginfig(thisfig);
  UseObject(sierpinski_gasket, Origin, (30,0,0), 1.5cm, 5);
endfig;
\end{verbatim}
\centerline{\XeTeXpdffile m3Dmanual-5.pdf}


\begin{verbatim}
let Fill = SolidFill;
beginfig(thisfig);
  save u; u=3cm;
  ObjectColor := (1, 215/255, 0);
  ObjectPath :=
  (eps, 0.65){right}
  ...{up}(0.45, 1)
  --(0.5, 1){down}
  ...{left}(0.1,0.6)% bowl
  {right}...{down}(0.15,0.55)
  ...{down}(0.075,0.35){down}
  ...{down}(0.075, 0.2)
  ...(0.15,0.15){down}
  ...{left}(0.1,0.1){right}
  ...{right}(0.4,0.05)
  --(0.4,0){left}
  ...{left}(eps,0.05);
  OnDepth;
    % contents
    Refpoint (0,0,0.8u);
    Action (ObjectColor := red;
      UseObject(revolution, (0, 0, 0), Origin, u,
  (eps, ypart point 0.8 of ObjectPath)..point 0.8 of ObjectPath);
      ObjectColor := (1, 215/255, 0););
    % bowl
    Refpoint (0,0,u);
    Action (UseObject(revolution, (0, 0, 0), Origin, u,
  subpath (0.75,3) of ObjectPath););
    % stem
    Refpoint (0,0,0.5u);
    Action (UseObject(revolution, (0, 0, 0), Origin, u,
  subpath (3,8) of ObjectPath));
    % foot
    Refpoint (0,0,0.25u);
    Action (UseObject(revolution, (0, 0, 0), Origin, u,
  subpath (8,11) of ObjectPath));
  endOnDepth;
endfig;
\end{verbatim}
\centerline{\XeTeXpdffile m3Dmanual-6.pdf}


\begin{verbatim}
Object molecule =
  M1 = (0, 0, 1); M2 = (1, 0, 0); M3 = (0, 1, 0);
  M4 = (-1, 0, 0); M5 = (0, -1, 0); M6 = (0, 0, -1);
  save srad, lrad, j; srad := 0.25; lrad := 0.1;
  OnDepth;
    for i = 1 upto 6:
      Refpoint M[i];
      Action(ObjectColor:=red;
        UseObject(sphere, M[i], Origin, srad));
    endfor
    for i = 1, 6:
      for j = 2, 3, 4, 5:
        Refpoint 0.5[M[i], M[j]];
        Action(ObjectColor := 0.375white;
          SpheresLink(M[i], M[j], srad, srad, lrad));
      endfor
    endfor
    for i = 2 upto 5:
      Refpoint 0.5[M[i], M[if i = 5: 2 else: i+1 fi]];
      Action(ObjectColor := 0.375white;
        SpheresLink(M[i], M[if i = 5: 2 else: i+1 fi], srad, srad, lrad));
    endfor
  endOnDepth;
endObject;

let Fill = SolidFill;
beginfig(thisfig);
  UseObject(molecule, Origin, Origin, 2cm);
endfig;
\end{verbatim}
\centerline{\XeTeXpdffile m3Dmanual-7.pdf}

\begin{verbatim}
let Fill = SolidWireFill;
ObjectColor := 0.5white;
PenColor := green;
%FinePlotFlag:=true;
beginfig(thisfig);
  interim prologues := 1;
  pickup thin.nib;
  Euler(0,0,0,0.2cm);
  Frame(-15 step 5 until 15)(-10 step 5 until 10)(-4 step 2 until 6);
  Plot3D("4cosd(180/3.14159*(x++y))*mexp(-(x++y)*50)", -15, 15, -10, 10);
  FrameMark (0,0,4);
  endFrame;
endfig;
\end{verbatim}
\centerline{\XeTeXpdffile m3Dmanual-8.pdf}

下面例子展示了在 ~{\tt m3Dplain.mp} 中定义的被称为 ~{\tt cylinderlike} 对
象的用法。

\centerline{\XeTeXpdffile m3Dmanual-9.pdf}

 它的特殊参数是一个 $xOy$-封闭路径以及柱形的高度。因此,前面的图形可以如下
得到。 
\begin{verbatim}
let Fill = SolidFill;
ObjectColor := (1, 215/255, 0);
beginfig(thisfig);
  UseObject(cylinderlike,(0,0,0),(-210,0,0),1cm,
    for i= 0 upto 4:
      2dir(i/5*360)--dir((i+0.5)/5*360)--
    endfor cycle, 1);
endfig;
\end{verbatim}



它相当简单。然后是 ~{\tt revolution} 对象的一个复杂例子。
\begin{verbatim}
beginfig(thisfig);
  interim prologues:=1;
  Euler(0,0,0,0.5cm);
  save h, dh, r; h=4; dh = 1; r=0.75;
  Frame(-4 step 2 until 4)
    (-4 step 2 until 4)
    (-4 step 2 until 4);
  Inside;
  % hyperboloide ॆ une nappe
  ObjectColor := (0, 215/255, 1);
  UseObject(revolution,Origin,Origin,1,
    (r*sqrt(h*h+1), h){-r*h/sqrt(h*h+1),-1}
    for y = h-dh step -dh until -h-eps:
      ...(r*sqrt(y*y+1), y){-r*y/sqrt(y*y+1), -1} endfor);
  % cone
  ObjectColor := (215/255, 1, 0);
  UseObject(revolution,Origin,Origin,1, (r*h, h)--(eps, 0)--(r*h,-h));
  % hyperboloide ॆ deux nappes
  ObjectColor := (1, 215/255, 0);
  for side = "Inside", "Outside":
    scantokens side;
    UseObject(revolution,Origin,Origin,1,
      reverse((eps, 1){right} for y = 1+dh step dh until h+eps:
    ...(r*sqrt(y*y-1), y){r, sqrt(y*y-1)/y} endfor));
  endfor
  UseObject(revolution,Origin,Origin,1,
    (eps, -1){right} for y = 1+dh step dh until h+eps:
      ...(r*sqrt(y*y-1), -y){r, -sqrt(y*y-1)/y} endfor);
  endFrame;
endfig;
\end{verbatim}
\centerline{\XeTeXpdffile m3Dmanual-10.pdf}

以及一个 Bezier 由控制序列 ~{\tt BezierPatch} 得到的分片曲面 (当且
仅当 
 $ ~{\tt tracingchoices} >0$  时才画出控制点。)
  \begin{verbatim}
tracingchoices:=1;
beginfig(thisfig);
  interim prologues := 1;
  interim CurrentScale := 1.5cm;
  BezierPatch(10,
    (-1,-1,-0.5), (-1,-0.3,0), (-1,0.3,0), (-1,1,-0.5),
    (-0.3,-1,0), (-0.3,-0.3,0.5), (-0.3,0.3,0.5), (-0.3,1,0),
    (0.3,-1,0), (0.3,-0.3,0.5), (0.3,0.3,0.5), (0.3,1,0),
    (1,-1,-0.5), (1,-0.3,0), (1,0.3,0), (1,1,-0.5));
endfig;
\end{verbatim}
\centerline{\XeTeXpdffile m3Dmanual-11.pdf}
 
 
 被称为 ~{\tt tube} 的对象可以非常简单地处理一些像画扭结的复杂情形: 
\begin{verbatim}
Ox:=(1,0,0);
Oy:=(0,1,0);
Oz:=(0,0,1);

beginfig(thisfig);
  UseObject(tube,Origin,Origin,0.5cm,
    "(cosd(t)+2cosd(2t), sind(t)-2sind(2t), 2sind(3t))",
    0.5, 360, 0, false, false);
endfig;
\end{verbatim}
\centerline{\XeTeXpdffile m3Dmanual-12.pdf}

 这就是为什么关于这个对象更为复杂的版本:~{\tt etube}(enhanced tube 增
 强的管道)被定义。
\begin{verbatim}
beginfig(thisfig);
  UseObject(etube,Origin,Origin,0.5cm,
    "(cosd(t)+2cosd(2t), sind(t)-2sind(2t), 2sind(3t))",
    0.5, 360, 0, false, false);
endfig;
\end{verbatim}
\centerline{\XeTeXpdffile m3Dmanual-13.pdf}
\end{document}